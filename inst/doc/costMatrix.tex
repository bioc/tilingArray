\documentclass[11pt]{article}

\newcommand{\Robject}[1]{\texttt{#1}}
\newcommand{\Rpackage}[1]{\textit{#1}}
\newcommand{\Rfunction}[1]{\textit{#1}}
\newcommand{\Rclass}[1]{\textit{#1}}
\usepackage{/home/huber/R/share/texmf/Sweave}

\begin{document}

%------------------------------------------------------------
\title{Calculation of the cost matrix}
%------------------------------------------------------------
\author{Wolfgang Huber}
\maketitle

\section{Problem statement and definitions}
Let $y_{nj}$ be the data value at position (genomic coordinate)
$n=1,\ldots,N$ for replicate array $j=1,\ldots,J$. Hence we have $J$
arrays and sequences of length $N$.  The goal of this note is to
describe an $O(NJ)$ algorithm to calculate the cost matrix of a
piecewise linear model for the segmentation of the $(1,\ldots,N)$ axis.
It is implemented in the function \textit{costMatrix} in the package
\textit{tilingArray}. The cost matrix is the input for a dynamic
programming algorithm that finds the optimal (least squares)
segmentation.


The cost matrix $G_{km}$ is the sum of squared residuals for a segment
from $m$ to $m+k-1$ (i.\,e.\ including $m+k-1$ but excluding $m+k$),
\begin{equation}\label{eq:costFun}
G_{km} := \frac{1}{J} \sum_{j=1}^J \sum_{n=m}^{m+k-1} 
\left(y_{nj} - \hat{\mu}_{km} \right)^2
\end{equation}
where $1\le m \le m+k-1\le N$ and $\hat{\mu}_{km}$ is the mean of that segment,
\begin{equation}\label{eq:defmu}
\hat{\mu}_{km} = \frac{1}{Jk} \sum_{j=1}^J \sum_{n=m}^{m+k-1} y_{nj}.
\end{equation}

\textit{Sidenote:} a perhaps more straightforward definition of a cost
matrix would be $\bar{G}_{m'\,m} = G_{(m'-m)\,m}$, the sum of squared
residuals for a segment from $m$ to $m'-1$. I use
version~(\ref{eq:costFun}) because it makes it easier to use the
condition of maximum segment length $(k<=k_{\mbox{\scriptsize max}})$, 
which I need to get the algorithm from complexity $O(N^2)$ to $O(N)$.

\newpage
\section{Algebra}
\begin{eqnarray}
\lefteqn{G_{km} = \frac{1}{J} \sum_{j=1}^J \sum_{n=m}^{m+k-1} 
\left(y_{nj} - \hat{\mu}_{km} \right)^2} \\
%
&=& \frac{1}{J} \left( 
\sum_{n,j} y_{nj}^2 - \frac{1}{Jk} \left(\sum_{n',j'} y_{n'j'}\right)^2
\right) \\
%
&=& \frac{1}{J} \sum_n q_n - 
\frac{1}{J^2k}\left(\sum_n r_n\right)^2 \label{Gbyqr}
\end{eqnarray}
with
\begin{eqnarray}
q_n&:=&\sum_j  y_{nj}^2 \\
r_n&:=&\sum_j  y_{nj} 
\end{eqnarray}
If \texttt{y} is an $N\times J$ matrix, then the $N$-vectors \texttt{q} 
and \texttt{r} can be obtained by
\begin{Sinput}
q = rowSums(y*y)
r = rowSums(y) 
\end{Sinput}
Now define
\begin{eqnarray}
c_\nu &=& \sum_{n=1}^{\nu}r_n\\
d_\nu &=& \sum_{n=1}^{\nu}q_n
\end{eqnarray}
which be obtained from
\begin{Sinput}
c = cumsum(r) 
d = cumsum(q) 
\end{Sinput}
then (\ref{Gbyqr}) becomes
\begin{equation}
\frac{1}{J}(d_{m+k-1}-d_{m-1}) - 
\frac{1}{J^2k}(c_{m+k-1}^2-c_{m-1}^2)
\end{equation}





%% \begin{equation}
%% A_{km} = \frac{1}{J}\left(\sum_{n=m}^{m+k-1}r_n\right)^2,
%% \end{equation}
%% then
%% \begin{eqnarray}
%% A_{1m} &=& \frac{r_n^2}{J} \\
%% A_{km} &=& \frac{1}{J}\left(
%% A_{k-1,m} + r_{m+k-1}^2 + 2 r_{m+k-1} 
%% \underbrace{\left( \sum_{n=m}^{m+k-2}r_n\right)}_{=:\;c_{m+k-2}-c_{m-1}}
%% \right)
%% \end{eqnarray}
%% where
%% Thus we can construct a recursion for $G_{km}$, which can be used to 
%% build it up row by row:
%% \begin{eqnarray}
%% G_{1m} &=& \frac{q_n}{J} - \frac{r_n^2}{J^2} \\
%% \end{eqnarray}


\end{document}


